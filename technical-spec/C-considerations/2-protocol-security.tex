\documentclass[../midgard.tex]{subfiles}
\graphicspath{{\subfix{../images/}}}
\begin{document}

\section{Protocol security}
\label{h:protocol-security}

\subsection{Malicious majority stake attack}
\label{h:malicious-majority-stake-attack}

A majority stake attack can allow the adversary to degrade the Liveness and Finality security properties of a blockchain protocol, if the adversary holds a sufficient percentage (e.g. 51\%) of the network's primary resource (e.g. hashing power for Bitcoin, ADA for Cardano).
In practical terms, this attack can allow the adversary to revert recently confirmed transactions and extract value from victims by preventing the reverted transactions from being restored to the ledger.

Midgard's consensus protocol is implemented entirely via Cardano L1 smart contracts.
This means that an attack to revert any transaction that evolves the state of these L1 smart contract is as difficult as an attack against Cardano's finality security property, against which Cardano's Ouroboros consensus protocol is highly resistant.
In simpler terms, Cardano's full ADA supply on L1 is the resource against which a majority stake attack must prevail to affect Midgard's L2 ledger.

Furthermore, every L2 transaction in Midgard is only confirmed after two consensus transactions occur on L1:
\begin{itemize}
  \item First, the L2 transaction is included in a block header committed to Midgard's state queue.
  \item Second, the block header is merged to Midgard's confirmed state after waiting in the queue for the maturity period.
\end{itemize}

Both of the above L1 transactions must be reverted to revert a confirmed L2 transaction in Midgard.
Due to the mandatory maturity period between the two L1 transactions, this amounts to a long-range attack to fork Cardano's blockchain before the maturity period began.
However, according to the Ouroboros consensus protocol, Cardano nodes will always reject any forks that diverge from their local chain by more than 2160 blocks, which is approximately 12 hours (with very tight confidence bounds).
As Midgard's maturity period protocol parameter will certainly be longer than 24 hours (see \cref{h:protocol-parameters}), it will be impossible to revert both consensus transactions on L1 for a confirmed L2 transaction on Midgard.

Suppose the adversary succeeds in an attack against Cardano's finality property, reverting the second consensus transaction for a Midgard L2 transaction.
This reversion will restore the block header to the beginning of the state queue, allowing it to be immediately re-merged to Midgard's confirmed state.
Moreover, no conflicting block header can be merged to the confirmed state in its place.
Thus, the adversary's attack has no practical effect other than slightly delaying the inevitable confirmation.

Suppose the adversary reverts the first consensus transaction for a Midgard L2 transaction.
This reversion will remove the block header from the state queue.
If the adversary is \emph{not} colluding with the current operator in Midgard, then the current operator will simply recommit the block header to the state queue, nullifying any effects of the adversary's attack.
If the adversary is colluding with the current operator, then the adversary's attack is redundant---the same effect is achieved if the current operator abuses its power to censor L2 transactions.
However, Midgard has safeguards in place against such abuse (see \cref{h:operator-abuse-of-power}).

\subsection{Operator abuse of power}
\label{h:operator-abuse-of-power}

Each operator has the exclusive privilege to commit blocks to the state queue and resolve nodes in the settlement queue during their assigned shifts (see \cref{h:time-model}), the duration of which is controlled by the \code{shift\_duration} protocol parameter (see \cref{h:protocol-parameters}).
The current operator has discretion over whether and when to commit blocks to the state queue and the contents of those blocks.
This discretion can be abused to censor L2 transactions, but Midgard's consensus protocol has safeguards in place against such abuse.

Midgard deposits and withdrawals are initiated via L1 smart contracts that assign definite inclusion times to them.
An operator block is invalid if it contains these inclusion times in its event interval but fails to include the associated deposit or withdrawal events.
This ensures that if operators continue committing blocks to Midgard's state queue, then they cannot ignore deposit and withdrawal events.

Midgard L2 transaction requests are typically submitted to operators via a publicly accessible API, and they can be ignored by operators.
However, any user can escalate his L2 transaction request by posting a transaction order on L1.
Similar to Midgard deposits and withdrawals, an L1 transaction order is assigned an inclusion time that guarantees its inclusion in a subsequent valid block.

If Midgard operators stop committing blocks at all to the state queue, then the inclusion times on their own cannot guarantee that deposits, withdrawals, and L2 transactions will be processed in a timely manner.
However, for this extreme case, Midgard's consensus protocol includes the escape hatch mechanism, which allows a special non-optimistic block to be appended to the state queue by a non-operator.
This block can include any deposits, withdrawals, and L2 transactions that are verified on L1 to comply with Midgard's ledger rules.
This ensures that user funds cannot be stranded on Midgard even if its operators entirely stop committing blocks.

\subsection{Replay attack}
\label{h:replay-attack}

In a replay attack, the adversary repeats or delays a valid message to confuse the network and achieve a malicious outcome.

Midgard's entire consensus protocol is implemented as a set of smart contracts on Cardano L1 that evolve via L1 transactions.
Each of these L1 transactions must spend at least one utxo, which means that Cardano's ledger rule against double-spending utxos prevent these transactions from being replayed.

However, Cardano's ledger rules do not directly see the contents corresponding to any hashes in the consensus protocol state on L1, such as the utxos, L2 transactions, deposits, and withdrawals in a block header.
Nonetheless, Midgard has comprehensive ledger rules against the various ways in which utxos, L2 transactions, deposits, and withdrawals can be duplicated within a block or across blocks.
Any operator that commits a block header to the state queue that contravenes these ledger rules forfeits his bond if the corresponding fraud proof is verified on L1 within the block's maturity period.

Another form of replay attack involves reusing the same confirmed deposit or withdrawal event in the settlement queue to absorb more deposited funds than necessary into Midgard's reserve or release more funds than necessary from it for a withdrawal.
Midgard prevents this by requiring an L1 utxo to be created on L1 for every deposit and withdrawal, which must be spent whenever the corresponding deposit is absorbed or withdrawal is paid out.

\end{document}

