\documentclass[../midgard.tex]{subfiles}
\graphicspath{{\subfix{../images/}}}
\begin{document}

\section{Block}
\label{h:block}

A block consists of a header hash, a header, and a block body:
\begin{equation*}
    \T{Block} \coloneq \left\{
    \begin{array}{ll}
        \T{header\_hash} : & \T{HeaderHash} \\
        \T{header} : & \T{Header} \\
        \T{block\_body} : & \T{BlockBody}
    \end{array} \right\}
\end{equation*}

The block body contains the block's transactions, deposits, and withdrawals, along with
the unspent outputs that result from applying the block's transition to the previous block's unspent outputs.
\begin{equation*}
    \T{BlockBody} \coloneq \left\{
    \begin{array}{ll}
        \T{utxos} : & \T{UtxoSet} \\
        \T{transactions} : & \T{TxSet} \\
        \T{deposits} : & \T{DepositSet} \\
        \T{withdrawals} : & \T{WithdrawalSet} \\
    \end{array} \right\}
\end{equation*}

\Cref{fig:block-transition} shows how the block's \code{utxos} are derived from the previous block's \code{utxos} by applying the block's events.
Withdrawals are applied before transactions, which are applied before deposits.

\begin{figure}[htb] % place the figure ’here’ or at the page top
    \centering % center the figure
    \begin{tikzpicture}[node distance=2cm]
        \node (s1) [initialstate] {\codeNC{prev\_utxos}};
        \node (s2) [finalstate, right of = s1, xshift=10cm]
            {\codeNC{utxos}};
        
        \path[every node/.style={font=\sffamily\normalsize}] [arrow]
            (s1.south east) edge [bend right = 30]
                 node [anchor = north, above, yshift = 0.5em] {\codeNC{withdrawals}} ($(s1.south east)!0.33!(s2.south west)$)
            ($(s1.south east)!0.33!(s2.south west)$) edge [bend right = 30]
                 node [anchor = north, above, yshift = 0.5em] {\codeNC{transactions}} ($(s1.south east)!0.66!(s2.south west)$)
            ($(s1.south east)!0.66!(s2.south west)$) edge [bend right = 30]
                 node [anchor = north, above, yshift = 0.5em] {\codeNC{deposits}} (s2.south west)
            (s1.north east) edge [bend left = 15]
                node [anchor = south, below, yshift = -0.5em] {Block transition} (s2.north west);
    \end{tikzpicture}
    \caption{A block's transition from a previous block's utxo set to a new utxo set.}
    \label{fig:block-transition}
\end{figure}

The block is what gets serialized and stored on Midgard's data availability layer.
During serialization, each of the block body's sets is serialized as a sequence of pairs, sorted in ascending order on the unique key of each element.

However, only the header hash and header are stored on Cardano L1.
This is sufficient because the header specifies Merkle Patricia Trie (MPT) root hashes for each of the sets in the block body.
Each of these root hashes can be verified onchain by streaming over the corresponding set's elements, hashing them, and iteratively calculating the root hash.

\Cref{fig:block-tx-mpt} shows an example MPT representation of a block's transactions.
The $(\T{TxId_i}, \T{MidgardTx_i})$ pairs are the data blocks of the tree, which are individually hashed to form the leaves $L_i$ of the tree.
The hashes of sibling leaves are concatenated and hashed to form their parent nodes.
The nodes $N_{ij}$ are formed by concatenating and hashing their children's hashes.
The \code{transactions\_root} is formed by concatenating and hashing its children's hashes.

\begin{figure}[htb]
\centering
\begin{tikzpicture}[
    level distance=1.5cm,
    sibling distance=5cm,
    edge from parent/.style={draw, -{Circle[open]}, thick},
    every node/.style={merklenode},
    level 1/.style={sibling distance=8cm},  % Increased distance for first level
    level 2/.style={sibling distance=4cm}   % Adjusted distance for second level
]

% Root node
  \node[merkleroot] {transactions\_root: $\mathcal{H}(N_{12} \mid\mid N_{34})$}
    child { node[merklenode] {$N_{12}: \mathcal{H}(L_1 \mid\mid L_2)$}
        child { node[merkleleaf] {$L_1: \mathcal{H}(D_1)$}
          child { node[merkledata] {$D_1$ \\ $(\T{TxId_1}, \T{MidgardTx_1})$} }
        }
        child { node[merkleleaf] {$L_2: \mathcal{H}(D_2)$}
          child { node[merkledata] {$D_2$ \\ $(\T{TxId_2}, \T{MidgardTx_2})$} }
        }
    }
    child { node[merklenode] {$N_{34}: \mathcal{H}(L_3 \mid\mid L_4)$}
        child { node[merkleleaf] {$L_3: \mathcal{H}(D_3)$}
          child { node[merkledata] {$D_3$ \\ $(\T{TxId_3}, \T{MidgardTx_3})$} }
        }
        child { node[merkleleaf] {$L_4: \mathcal{H}(D_4)$}
          child { node[merkledata] {$D_4$ \\ $(\T{TxId_4}, \T{MidgardTx_4})$} }
        }
    };

\end{tikzpicture}
\caption{A Merkle Patricia Trie example for a block's transactions.}
\label{fig:block-tx-mpt}
\end{figure}

\subsection{Block header}
\label{h:block-header}

A block header is a record with fixed-size fields: integers, hashes, and fixed-size bytestrings.
A block header hash is 28 bytes in size and calculated via the Blake2b-224.
\begin{equation*}
\begin{split}
  \T{HeaderHash} &\coloneq \mathcal{H}_\T{Blake2b-224}(\T{Header}) \\
  \T{Header} &\coloneq \left\{
    \begin{array}{ll}
        \T{prev\_utxos\_root} : & \T{RootHash} \\
        \T{utxos\_root} : & \T{RootHash} \\
        \T{transactions\_root} : & \T{RootHash} \\
        \T{deposits\_root} : & \T{RootHash} \\
        \T{withdrawals\_root} : & \T{RootHash} \\
        \T{start\_time} : & \T{PosixTime} \\
        \T{end\_time} : & \T{PosixTime} \\
        \T{prev\_header\_hash} : & \T{HeaderHash} \\
        \T{operator\_vkey} : & \T{VerificationKey} \\
        \T{protocol\_version} : & \T{Int} \\
    \end{array} \right\}
\end{split}
\end{equation*}

These header fields are interpreted as follows:
\begin{itemize}
    \item The \code{*\_root} fields are the MPT root hashes of the corresponding sets in the block body.
    \item The \code{prev\_utxos\_root} is a copy of the \code{utxos\_root} from the previous block, included for convenience in the fraud proof verification procedures.
    \item The \code{start\_time} and \code{end\_time} fields are the block's event interval bounds (see \cref{h:time-model}).
    \item The \code{prev\_header\_hash} field is a hash of the previous block header. For the genesis block, this field is set to 28 \code{0x00} bytes.
    \item The \code{operator\_vkey} field is the cryptographic verification key for signatures of the operator who committed the block header to the L1 state queue.
    \item The \code{protocol\_version} is the Midgard protocol version that applies to this block.
\end{itemize}


\end{document}
